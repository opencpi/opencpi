\iffalse
This file is protected by Copyright. Please refer to the COPYRIGHT file
distributed with this source distribution.

This file is part of OpenCPI <http://www.opencpi.org>

OpenCPI is free software: you can redistribute it and/or modify it under the
terms of the GNU Lesser General Public License as published by the Free Software
Foundation, either version 3 of the License, or (at your option) any later
version.

OpenCPI is distributed in the hope that it will be useful, but WITHOUT ANY
WARRANTY; without even the implied warranty of MERCHANTABILITY or FITNESS FOR A
PARTICULAR PURPOSE. See the GNU Lesser General Public License for more details.

You should have received a copy of the GNU Lesser General Public License along
with this program. If not, see <http://www.gnu.org/licenses/>.
\fi
\def\snippetpath{../../../../../../doc/av/tex/snippets}
% Usage:
% \def\snippetpath{../../../../../doc/av/tex/snippets/}
% % Usage:
% \def\snippetpath{../../../../../doc/av/tex/snippets/}
% % Usage:
% \def\snippetpath{../../../../../doc/av/tex/snippets/}
% \input{\snippetpath/includes}
% From then on, you can use "input" With no paths to get to "snippets"
% You also get all "major" snippets not part of the global LaTeX_Header
% NOTE: If not using the global LaTeX_Header, you need to
% \usepackage{ifthen} to use the \githubio macro

\hyphenation{ANGRY-VIPER} % Tell it where to hyphenate
\hyphenation{Cent-OS} % Tell it where to hyphenate
\hyphenation{install-ation} % Tell it where to hyphenate

\newcommand{\todo}[1]{\textcolor{red}{TODO: #1}\PackageWarning{TODO:}{#1}} % To do notes
\newcommand{\code}[1]{\texttt{#1}} % For inline code snippet or command line
\newcommand{\sref}[1]{Section~\ref{#1}} % To quickly reference a section

% To quickly reference a versioned PDF on github.io
\def\ocpiversion{1.5.0}

% This gives a link to github.io document. By default, it puts the filename.
% You can optionally change the link, e.g.
% \githubio{FPGA\_Vendor\_Tools\_Installation\_Guide.pdf} vs.
% \githubio[\textit{FPGA Vendor Tools Installation Guide}]{FPGA\_Vendor\_Tools\_Installation\_Guide.pdf}
% or if you want the raw ugly URL to come out, \githubioURL{FPGA_Vendor_Tools_Installation_Guide.pdf}
\newcommand{\githubio}[2][]{% The default is for FIRST param!
\href{http://opencpi.github.io/releases/\ocpiversion/#2}{\ifthenelse{\equal{#1}{}}{\texttt{#2}}{#1}}}
\newcommand{\githubioURL}[1]{\url{http://opencpi.github.io/releases/\ocpiversion/#1}}
% Lastly, if you want a SINGLE leading path stripped, e.g. assets/X.pdf => X.pdf:
\newcommand{\githubioFlat}[1]{%
\StrBehind{#1}{/}[\den]%
\href{http://opencpi.github.io/releases/\ocpiversion/#1}{\texttt{\den}}%
}

% Fix import paths
\makeatletter
\def\input@path{{\snippetpath/}}
\makeatother

% From then on, you can use "input" With no paths to get to "snippets"
% You also get all "major" snippets not part of the global LaTeX_Header
% NOTE: If not using the global LaTeX_Header, you need to
% \usepackage{ifthen} to use the \githubio macro

\hyphenation{ANGRY-VIPER} % Tell it where to hyphenate
\hyphenation{Cent-OS} % Tell it where to hyphenate
\hyphenation{install-ation} % Tell it where to hyphenate

\newcommand{\todo}[1]{\textcolor{red}{TODO: #1}\PackageWarning{TODO:}{#1}} % To do notes
\newcommand{\code}[1]{\texttt{#1}} % For inline code snippet or command line
\newcommand{\sref}[1]{Section~\ref{#1}} % To quickly reference a section

% To quickly reference a versioned PDF on github.io
\def\ocpiversion{1.5.0}

% This gives a link to github.io document. By default, it puts the filename.
% You can optionally change the link, e.g.
% \githubio{FPGA\_Vendor\_Tools\_Installation\_Guide.pdf} vs.
% \githubio[\textit{FPGA Vendor Tools Installation Guide}]{FPGA\_Vendor\_Tools\_Installation\_Guide.pdf}
% or if you want the raw ugly URL to come out, \githubioURL{FPGA_Vendor_Tools_Installation_Guide.pdf}
\newcommand{\githubio}[2][]{% The default is for FIRST param!
\href{http://opencpi.github.io/releases/\ocpiversion/#2}{\ifthenelse{\equal{#1}{}}{\texttt{#2}}{#1}}}
\newcommand{\githubioURL}[1]{\url{http://opencpi.github.io/releases/\ocpiversion/#1}}
% Lastly, if you want a SINGLE leading path stripped, e.g. assets/X.pdf => X.pdf:
\newcommand{\githubioFlat}[1]{%
\StrBehind{#1}{/}[\den]%
\href{http://opencpi.github.io/releases/\ocpiversion/#1}{\texttt{\den}}%
}

% Fix import paths
\makeatletter
\def\input@path{{\snippetpath/}}
\makeatother

% From then on, you can use "input" With no paths to get to "snippets"
% You also get all "major" snippets not part of the global LaTeX_Header
% NOTE: If not using the global LaTeX_Header, you need to
% \usepackage{ifthen} to use the \githubio macro

\hyphenation{ANGRY-VIPER} % Tell it where to hyphenate
\hyphenation{Cent-OS} % Tell it where to hyphenate
\hyphenation{install-ation} % Tell it where to hyphenate

\newcommand{\todo}[1]{\textcolor{red}{TODO: #1}\PackageWarning{TODO:}{#1}} % To do notes
\newcommand{\code}[1]{\texttt{#1}} % For inline code snippet or command line
\newcommand{\sref}[1]{Section~\ref{#1}} % To quickly reference a section

% To quickly reference a versioned PDF on github.io
\def\ocpiversion{1.5.0}

% This gives a link to github.io document. By default, it puts the filename.
% You can optionally change the link, e.g.
% \githubio{FPGA\_Vendor\_Tools\_Installation\_Guide.pdf} vs.
% \githubio[\textit{FPGA Vendor Tools Installation Guide}]{FPGA\_Vendor\_Tools\_Installation\_Guide.pdf}
% or if you want the raw ugly URL to come out, \githubioURL{FPGA_Vendor_Tools_Installation_Guide.pdf}
\newcommand{\githubio}[2][]{% The default is for FIRST param!
\href{http://opencpi.github.io/releases/\ocpiversion/#2}{\ifthenelse{\equal{#1}{}}{\texttt{#2}}{#1}}}
\newcommand{\githubioURL}[1]{\url{http://opencpi.github.io/releases/\ocpiversion/#1}}
% Lastly, if you want a SINGLE leading path stripped, e.g. assets/X.pdf => X.pdf:
\newcommand{\githubioFlat}[1]{%
\StrBehind{#1}{/}[\den]%
\href{http://opencpi.github.io/releases/\ocpiversion/#1}{\texttt{\den}}%
}

% Fix import paths
\makeatletter
\def\input@path{{\snippetpath/}}
\makeatother

\input{LaTeX_Header}
\usepackage{graphicx}
\graphicspath{ {figures/} }
\usepackage{fancyhdr}
\usepackage{colortbl}
%\usepackage[margin=.75in]{geometry}
\usepackage[justification=centering]{caption}
\pagestyle{fancy}
\lhead{Zipper/Myriad-RF 1 Daughtercards}
\rhead{ANGRYVIPER Team}
\definecolor{drkgreen}{rgb}{0,.6,0}
\renewcommand{\headrulewidth}{0pt}
%\newcommand{\code}[1]{\texttt{#1}} % For inline code snippet or command line
\definecolor{blue}{rgb}{.7,.8,.9}
\begin{document}
\section*{Deprecation Notice:}
Beginning with OpenCPI Version 1.5, support for Lime Microsystems' Zipper card is now deprecated. This document will not be updated any further.
\section*{Limitations: Myriad-RF 1 and Zipper Daughtercard}
The following is a list of the known limitations with ANGRYVIPER and the Myriad-RF 1 and Zipper Daughtercard:\par
	\begin{itemize}
	\item[1)] The Zipper Carrier card is no longer being produced by the manufacturer. The card is Open Source so you are able to fabricate one yourself.
	\item[2)] Sample Rate Limit on ML605, Stratix IV and ZedBoard Rev D
	\item[3)] Clock initialization error with SI5351 proxy RCC component
	\item[4)] Only VADJ supported on ZedBoard is 2.5 V
	\end{itemize}
This document will describe the limitations and any workarounds.
\subsection*{Sample Rate Limitations}
During testing of the ANGRYVIPER reference applications, data fidelity issues have been observed in the following scenarios:
%\begin{flushleft}
%\begin{center}
\begin{table}[H]

		\label{table:samplelimits}
		\begin{tabularx}{\textwidth}{|c|X|}
			\hline
			\rowcolor{blue}
			\textbf{Platform} & \textbf{Maximum Sample Rate} \\
			\hline
			ML605 & 34 MS/s\\
			\hline
			Stratix IV & 25 MS/s\\
			\hline
			ZedBoard Rev D & 34 MS/s\\
		    \hline
		\end{tabularx}
		\caption{Sample rate limitations per platform}

		\textit{Note: } ZedBoard Rev C does not show any such limitations. The Matchstiq-Z1 and Picoflexor T6A-S1 do not use use the Zipper daughtercard and therefore do not show any such limitations.
\end{table}

%\end{center}
%\begin{enumerate}
%\item ZedBoard Rev D
%\end{enumerate} certain sample rates on Rev D of the ZedBoard platform, the ALST4 and ML605. Specifically, the FSK application did not run correctly above 36 MS/s. Other ANGRYVIPER platforms (Matchstiq Z1, ML605, ALST4, ZedBoard Rev C) do not show this limitation.\par\medskip
\noindent As of the 1.2 release of ANGRYVIPER, it is suspected that these upper sampling rate data fidelity problems are caused by phase incoherence due to the non-source-synchronous ADC/DAC clock source on the Zipper that are not being accounted for. This will be investigated during future release(s).
%As of the 1.1 release of ANGRYVIPER, there are no card specific constraints in the UCFs for these platforms. It is suspected that properly constraining the interface between the ML605/Stratix IV/ZedBoard and the daughtercard would increase the sample rate limit.
\subsection*{Clock Initialization Error with SI5351 Proxy RCC Component}
When using ANGRYVIPER and the Myriad-RF 1 and Zipper Daughtercard, the SI5351 proxy component experiences the following intermittent errors:\par\medskip
\noindent \code{Exception thrown: Code 0x17, level 0, error: 'Worker 'clock\_gen' produced error during the 'stop' control operation: SI5351 has not completed system initialization'}\par\medskip
\noindent \code{Exception thrown: Code 0x17, level 0, error: 'Worker 'clock\_gen' produced error during the 'initialize' control operation: SI5351 has not completed system initialization'}\par\medskip
\noindent Per an application note of the SI5351\cite{an_619}, this error means that the IC is in 'System Initialization Mode'. It isn't recommended to read or write registers during this period, so the proxy correctly throws an exception. This mode is only supposed to be encountered during power up, but the error occurs intermittently  during use of this platform.\par\medskip
\noindent There is no known workaround for this error. The error does not typically occur multiple times in a row, so re-running the application after an occurrence typically completes successfully.
\subsection*{Only VADJ supported on ZedBoard is 2.5 V}
The ZedBoard platform  and the Myriad-RF 1 and Zipper Daughtercard support three different Vadj voltages for the FMC LPC connector. During testing of the ANGRYVIPER reference applications, the only working configuration was 2.5 V. 3.3 V was not tested and 1.8 V showed data fidelity issues.  It is recommended that 2.5 V be used.
\pagebreak
  \begin{thebibliography}{1}

  \bibitem{an_619} AN 619: Manually Generating an Si5351 Register Map\\
  https://www.silabs.com/Support Documents/TechnicalDocs/AN619.pdf

  \end{thebibliography}
\end{document}
