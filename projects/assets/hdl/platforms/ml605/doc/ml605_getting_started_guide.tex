\documentclass{article}
\iffalse
This file is protected by Copyright. Please refer to the COPYRIGHT file
distributed with this source distribution.

This file is part of OpenCPI <http://www.opencpi.org>

OpenCPI is free software: you can redistribute it and/or modify it under the
terms of the GNU Lesser General Public License as published by the Free Software
Foundation, either version 3 of the License, or (at your option) any later
version.

OpenCPI is distributed in the hope that it will be useful, but WITHOUT ANY
WARRANTY; without even the implied warranty of MERCHANTABILITY or FITNESS FOR A
PARTICULAR PURPOSE. See the GNU Lesser General Public License for more details.

You should have received a copy of the GNU Lesser General Public License along
with this program. If not, see <http://www.gnu.org/licenses/>.
\fi

% TODO: Version numbers?
\usepackage{graphicx}
\graphicspath{ {figures/} }
\usepackage{fancyhdr}
\usepackage{colortbl}
\usepackage[margin=.75in]{geometry}
\usepackage{hyperref}
\usepackage{listings}
\pagestyle{fancy}
\lhead{Board Support Package Documentation}
\rhead{ANGRYVIPER Team}
\renewcommand{\headrulewidth}{0pt}
\newcommand{\shellcmd}[1]{\texttt{\$ #1\\}}
\newcommand{\terminaloutput}[1]{\texttt{#1}}
\definecolor{blue}{rgb}{.5,1,1}
\definecolor{drkgreen}{rgb}{0,.6,0}
\begin{document}
\section*{ML605 Getting Started Guide}
\subsection*{Hardware Prerequisites}
This section describes the hardware prerequisites required for an operational Xilinx Virtex-6 ML605 platform using ANGRYVIPER. Note that the slot configurations in Table \ref{table:supported_slots} are limited by what FPGA bitstreams are currently built by ANGRYVIPER and not by what hardware configurations are theoretically possible using ANGRYVIPER.\\ \\
Hardware prerequisites are as follows.
\begin{itemize}
\item An ML605 board, which has undergone an ANGRYVIPER-specific initial one-time hardware setup \cite{ml605_hardware_setup} and is plugged into a PCIE slot of an x86 computer.
\item Optionally, one of the following FMC card configurations in Table  \ref{table:supported_slots} may exist
\end{itemize}
\begin{center}
        \begin{table}[!htbp]
        \centering
        \caption{ANGRYVIPER-supported ML605 hardware FMC slot configurations}
        \label{table:supported_slots}
        \begin{tabular}{c|c|c|}
                \cline{2-3}
                & FMC LPC slot & FMC HPC slot \\ \hline
                \multicolumn{1}{|c|}{Zipper A setup} & Modified\cite{zipper_mods} Zipper/MyriadRF & (empty)\\
                \multicolumn{1}{|c|}{ } & transceiver card & \\ \hline
                \multicolumn{1}{|c|}{Zipper B setup} & (empty) & Modified\cite{zipper_mods} Zipper/MyriadRF \\
                \multicolumn{1}{|c|}{ } & & transceiver card \\ \hline
        \end{tabular}
        \end{table}
\end{center}

\subsection*{Software Prerequisites}
\begin{itemize}
\item A CentOS 6 or CentOS 7 operating system installed on the x86 computer.
\item Xilinx ISE installed on the x86 computer, including the necessary Xilinx cable driver modifications necessary for CentOS. For information on these modifications, refer to \cite{angryviper_fpga_vendor_tool_guide}
\item ANGRYVIPER framework and prerequisite RPMs installed on the x86 computer. For more information refer to \cite{angryviper_installation_guide}
\item OpenCPI Base Project compiled for ml605. For more information refer to the "Base Project" section in \cite{angryviper_installation_guide}
\item ANGRYVIPER assets project compiled for ml605. For more information refer to the "ANGRYVIPER assets" section in \cite{angryviper_installation_guide}
\end{itemize}

\subsection*{Proof of Operation}
The following commands may be run in order to verify correct ANGRYVIPER operation on the x86/ML605 system.\\ \\
Existence of ML605 RCC/HDL containers may be verified by running the following command and verifying that similar output is produced.\\
\noindent\terminaloutput{\$ ocpirun -C\\
Available containers:\\
 \#  Model\hspace{6ex} Platform\hspace{3ex}    OS\hspace{5ex}     OS-Version\hspace{1ex}  Arch\hspace{3ex} Name\\
 0  hdl\hspace{9ex}   ml605\hspace{39ex} PCI:0000:02:00.0\\
 1  rcc\hspace{9ex}   centos7\hspace{4ex}      linux\hspace{2ex}  c7\hspace{10ex}          x86\_64\hspace{1ex} rcc0\\
}\\ \\
Operation of the RCC container can be verified by running the hello application via the following command and verifying that identical output is produced. Note that the OCPI\_LIBRARY\_PATH environment variable must be setup correctly for your system prior to running this command.\\
\noindent\terminaloutput{\$ ocpirun -t 1 \$OCPI\_PROJECT\_PATH/examples/xml/hello.xml \\
Hello, world} \\ \\
Simultaneous RCC/HDL container operation can be verified by running the testbias application via the following command and verifying that identical output is produced. Note that the OCPI\_LIBRARY\_PATH environment variable must be setup correctly for your system prior to running this command.\\
\noindent\terminaloutput{ocpirun -d -m bias=hdl testbias.xml \\
Property  0: file\_read.fileName = "test.input" (cached)\\
Property  1: file\_read.messagesInFile = "false" (cached)\\
Property  2: file\_read.opcode = "0" (cached)\\
Property  3: file\_read.messageSize = "16"\\
Property  4: file\_read.granularity = "4" (cached)\\
Property  5: file\_read.repeat = "<unreadable>"\\
Property  6: file\_read.bytesRead = "0"\\
Property  7: file\_read.messagesWritten = "0"\\
Property  8: file\_read.suppressEOF = "false"\\
Property  9: file\_read.badMessage = "false"\\
Property 10: file\_read.ocpi\_debug = "false" (parameter)\\
Property 11: file\_read.ocpi\_endian = "little" (parameter)\\
Property 12: bias.biasValue = "16909060" (cached)\\
Property 13: bias.ocpi\_debug = "false" (parameter)\\
Property 14: bias.ocpi\_endian = "little" (parameter)\\
Property 15: bias.test64 = "0"\\
Property 16: file\_write.fileName = "test.output" (cached)\\
Property 17: file\_write.messagesInFile = "false" (cached)\\
Property 18: file\_write.bytesWritten = "0"\\
Property 19: file\_write.messagesWritten = "0"\\
Property 20: file\_write.stopOnEOF = "true" (cached)\\
Property 21: file\_write.ocpi\_debug = "false" (parameter)\\
Property 22: file\_write.ocpi\_endian = "little" (parameter)\\
Property  3: file\_read.messageSize = "16"\\
Property  5: file\_read.repeat = "<unreadable>"\\
Property  6: file\_read.bytesRead = "4000"\\
Property  7: file\_read.messagesWritten = "251"\\
Property  8: file\_read.suppressEOF = "false"\\
Property  9: file\_read.badMessage = "false"\\
Property 15: bias.test64 = "0"\\
Property 18: file\_write.bytesWritten = "4000"\\
Property 19: file\_write.messagesWritten = "250"\\
\\
}

\pagebreak
  \begin{thebibliography}{1}


  \bibitem{ml605_hardware_setup} ML605 Hardware Setupe\\
	 ml605\_hardware\_setup.pdf
  \bibitem{angryviper_fpga_vendor_tool_guide} ANGRYVIPER FPGA Vendor Tools Guide\\
	 AngryViper\_FPGA\_Vendor\_Tools\_Guide.pdf
	   \bibitem{angryviper_installation_guide} ANGRYVIPER Installation Guide\\
	 AngryViper\_Installation\_Guide.pdf
	   \bibitem{zipper_mods} Required Modifications for Myriad-RF 1 and Zipper Daughtercards\\
	 Required\_Modifications\_for\_Myriad-RF\_1\_Zipper\_Daughtercards.pdf

  \end{thebibliography}

\end{document}
