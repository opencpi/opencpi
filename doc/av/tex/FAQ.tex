\iffalse
This file is protected by Copyright. Please refer to the COPYRIGHT file
distributed with this source distribution.

This file is part of OpenCPI <http://www.opencpi.org>

OpenCPI is free software: you can redistribute it and/or modify it under the
terms of the GNU Lesser General Public License as published by the Free Software
Foundation, either version 3 of the License, or (at your option) any later
version.

OpenCPI is distributed in the hope that it will be useful, but WITHOUT ANY
WARRANTY; without even the implied warranty of MERCHANTABILITY or FITNESS FOR A
PARTICULAR PURPOSE. See the GNU Lesser General Public License for more details.

You should have received a copy of the GNU Lesser General Public License along
with this program. If not, see <http://www.gnu.org/licenses/>.
\fi
%----------------------------------------------------------------------------------------
% Update the docTitle and docVersion per document
%----------------------------------------------------------------------------------------
\def\docTitle{Frequently Asked Questions}
\def\docVersion{1.2}
%----------------------------------------------------------------------------------------
\input{snippets/LaTeX_Header.tex}
\setlength{\parindent}{0pt} % Don't indent all paragraphs
\newcommand{\forceindent}{\leavevmode{\parindent=1em\indent}}
\date{Version \docVersion} % Force date to be blank and override date with version
\title{\docTitle}
\lhead{\small{\docTitle}}
\usepackage{enumitem}
%----------------------------------------------------------------------------------------
\begin{document}
\maketitle
\thispagestyle{fancy}
\newpage

        \begin{center}
        \textit{\textbf{Revision History}}
                \begin{table}[H]
                \label{table:revisions} % Add "[H]" to force placement of table
                        \begin{tabularx}{\textwidth}{|c|X|l|}
                        \hline
                        \rowcolor{blue}
                        \textbf{Revision} & \textbf{Description of Change} & \textbf{Date} \\
                        \hline
                        v1.1 & Initial creation for OpenCPI 1.1 & 3/2017 \\
                        \hline
                        v1.2 & Updated for OpenCPI Release 1.2 & 8/2017 \\
		                \hline
                        \end{tabularx}
                \end{table}
        \end{center}
\newpage
% How to add a new question / answer:
% \item[Question]~\\
% Answer
\section{General Questions}
% AV-1724
\begin{description}[style=nextline]
\item[Is the RPM suite a standalone install?]~\\
\label{faq:whatis}%
Yes, the RPMs distributed by the ANGRYVIPER team incorporates and extends the Free / Open Source Project ``OpenCPI.'' Any OpenCPI installation documents that still exist are for reference and legacy users. All other OpenCPI documentation still applies and should be referenced. Do \textbf{not} attempt to install OpenCPI from source at the same time as the RPM distribution.
\end{description}

\section{Install-Specific Questions}
\begin{description}[style=nextline]
% AV-1724
\item[Does it matter what version of CentOS is used?]~\\
Both CentOS~6 and CentOS~7 are supported as long as the proper version of the RPM is used. Local hardware support (\textit{e.g.} PCIe-based platforms) is officially supported on both OS releases starting with Version 1.1.
\end{description}

\section{General Usage Questions}
\begin{description}[style=nextline]
% AV-1724
\item[Make error: ``*** isim not an available HDL platform.  Stop.'']~\\
Either the Base Project was never built, or the variable \path{OCPI_PROJECT_PATH} is not set. This is explained in the \textit{Getting Started Guide}.

% 2016-10-19
\item[I am trying to run a demo application with ``ocpirun'' and artifacts are not being found.]~\\
The usual causes of this are:
\begin{itemize}
\setlength\itemsep{0pt}
\item Base Project was not built for the target platform
\begin{itemize}
\item Consult the \textit{Getting Started Guide}
\end{itemize}
\item \path{OCPI_LIBRARY_PATH} was not properly set
\begin{itemize}
\item View the artifacts being checked by adding ``\code{-l 8}'' on the \texttt{ocpirun} command line to increase the logging level
\end{itemize}
\end{itemize}
\end{description}

\section{Xilinx-Specific Questions}
\begin{description}[style=nextline]
% AV-1724, 2016-10-15
\item[Are there any other setups I need to perform on the Xilinx Vivado or ISE side?]~\\
No, we abstract away a lot of the requirements if you simply install it in \path{/opt/Xilinx} and point the setup variables to it (see \path{/opt/opencpi/cdk/env.d/xilinx.sh.example} and the \textit{RPM Installation Guide}).

% AV-1736
Additionally, importing the Xilinx setup scripts, \textit{e.g.} ``\path{source /opt/Xilinx/14.7/ISE_DS/settings64.sh}'' or ``\path{source /opt/Xilinx/Vivado/2017.1/settings64.sh}'', can cause other problems and \textbf{should not be performed}.

% 2016-10-07
\item[The ZedBoard comes with a license, but it is for the Vivado tools.]~\\
Xilinx's ``WebPack'' Vivado or ISE license is sufficient to do anything with the ZedBoard.

\textit{ISE Note:} As for purchasing, you can ``rollback'' a Vivado license by contacting Xilinx and they will issue you an ISE license with the same expiration with a gentleman's agreement that you won't use both at the same time.
\end{description}

\end{document}
