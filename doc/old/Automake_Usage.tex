\documentclass{article}
\iffalse
This file is protected by Copyright. Please refer to the COPYRIGHT file
distributed with this source distribution.

This file is part of OpenCPI <http://www.opencpi.org>

OpenCPI is free software: you can redistribute it and/or modify it under the
terms of the GNU Lesser General Public License as published by the Free Software
Foundation, either version 3 of the License, or (at your option) any later
version.

OpenCPI is distributed in the hope that it will be useful, but WITHOUT ANY
WARRANTY; without even the implied warranty of MERCHANTABILITY or FITNESS FOR A
PARTICULAR PURPOSE. See the GNU Lesser General Public License for more details.

You should have received a copy of the GNU Lesser General Public License along
with this program. If not, see <http://www.gnu.org/licenses/>.
\fi
\usepackage{fancyhdr}
\usepackage{colortbl}
\usepackage{scrextend}
\usepackage{hyperref}
\setcounter{secnumdepth}{0} % No numbers, but import into TOC anyway
\usepackage{listings}
% \begin{lstlisting}
\usepackage[margin=.75in]{geometry}
\usepackage{microtype}
\usepackage{indentfirst} % Indent all paragraphs
\pagestyle{fancy}
\headheight=23pt
\lhead{Technical Note:\\Automake Usage}
\chead{\textbf{INTERNAL USE ONLY\\
NOT FOR DISTRIBUTION}}
\cfoot{\textbf{INTERNAL USE ONLY}}
\renewcommand{\headrulewidth}{0pt}
\definecolor{blue}{rgb}{.5,1,1}

% This block is to make sure there is 3cm min at the bottom of a page before a new section or subsection is allowed to start. Otherwise, next page.
% Modified from http://tex.stackexchange.com/a/152278
\usepackage{etoolbox}
\newskip\mfilskip
\mfilskip=0pt plus 3cm\relax
\newcommand{\mfilbreak}{\vspace{\mfilskip}\penalty -200%
  \ifdim\lastskip<\mfilskip\vspace{-\lastskip}\else\vspace{-\mfilskip}\fi}
\pretocmd{\section}{\mfilbreak}{}{}
\pretocmd{\subsection}{\mfilbreak}{}{}
% end [sub]section pushes

\usepackage{tabularx}
% These define tabularx columns "C" and "R" to match "X" but center/right aligned
\newcolumntype{C}{>{\centering\arraybackslash}X}
\newcolumntype{R}{>{\raggedleft\arraybackslash}X}

\newcommand{\todo}[1]{\textcolor{red}{TODO: #1}\PackageWarning{TODO:}{#1}}
% \parindent=20pt
% \hangindent=0.7cm
\begin{document}


% \tableofcontents
% \vspace{1pc}
% \hrule
\section{Document Scope}
This document is a Technical Note that describes the building of OpenCPI and its components in the ``develop'' branch since moving to \texttt{autotools}.

\section{Overview}
Historically, OpenCPI development has been using multiple hand-written \texttt{Makefile}s that did not allow for parallelism and the build logic was spread in many unfindable locations. It also required up to \textit{thirty} environmental variables. This document is for users and developers who wish to \textit{continue} to use the legacy build system instead of RPM-based builds.

\section{Backwards Compatibility}
The \texttt{ocpi-configure} and \texttt{cross-configure} scripts capture the various environmental variables set and import them into the rest of the build system. \textbf{This script should be executed \textit{every} time environmental variables are updated, especially if changing RCC targets.} This script also generates the \texttt{Makefile}.

\subsection{Environmental Variables}
\todo{This entire section is likely obsolete, e.g. OCPI\_CROSS\_HOST has moved to OCPI\_PLATFORM\_TARGET in v1.1... - adm, Jan 2017}\\
The classic OpenCPI environment uses about thirty variables, usually set with scripts called in the \texttt{env} subdirectory. Since most of the build infrastructure is the same, these variables are pre-set by the RPM installation, but the user can continue to override them if they require it. The variables that the user \textit{might} still concern themselves with include:

\begin{center}
\begin{tabularx}{\textwidth}{|r|X|}
\hline
\rowcolor{blue}
Variable & Description \\ \hline
\hfill\verb+OCPI_CDK_DIR+\footnote{\label{env_footnote}These variables are automatically set with shell login once the main RPM has been installed.} & Global CDK location \\ \hline
\hfill\verb+OCPI_CROSS_HOST+ & Cross-compilation target (e.g. \texttt{arm-xilinx-linux-gnueabi}) \\ \hline
\hfill\verb+OCPI_CROSS_BUILD_BIN_DIR+\footnote{You can often use the \texttt{locate} command to find the \verb+${OCPI_CROSS_HOST}+-ranlib executable.} & Cross-compilation GCC chain location\newline (e.g. \verb+/opt/Xilinx/14.7/ISE_DS/EDK/gnu/arm/lin/bin/+) \\ \hline
\hfill\verb+OCPI_PROJECT_PATH+\footref{env_footnote} & A colon-separated list of other projects to include when building the current project. Must always include the ``Base Project'', even when building the Base Project itself. \\ \hline
\hfill\verb+OCPI_TARGET_KERNEL_DIR+ & Kernel headers directory for driver compilation\newline(e.g. \verb+<path>/platforms/zed/release/kernel-headers+) \\ \hline
\hfill\verb+OCPI_TARGET_PLATFORM+ & Target platform (e.g. \texttt{xilinx13\_4})\\ \hline
\hfill\verb+OCPI_TOOL_HOST+\footref{env_footnote} & Host system type (e.g. \verb+linux-c7-x86_64+) \\ \hline
\hfill\verb+OCPI_TOOL_PLATFORM+ & Tool/host platform (\textit{seldom required}, e.g. \texttt{centos7})\\ \hline
\hfill\verb+OCPI_{TOOLNAME}_DIR+ & The base installation directory for the \{Altera,ModelSim,Xilinx\} tools\newline (e.g. \texttt{/opt/Xilinx/})\\ \hline
\hfill\verb+OCPI_{TOOLNAME}_LICENSE_FILE+ & The location of the tool's licensing file or server. The end user should \textbf{not} directly set \verb+LM_LICENSE_FILE+.\\ \hline
\hfill\verb+OCPI_{TOOLNAME}_VERSION+ & The version of \{Altera,Xilinx\} tools installed (e.g. \texttt{14.7})\\ \hline
\end{tabularx}
\end{center}

Any other \textit{unset} variables are set to a reasonable default, and a warning is given upon execution indicating what values were chosen.

\subsection{Building}
\subsubsection{Native RCC}
\begin{itemize}
\item \verb+./ocpi-configure && make -j rcc+
\end{itemize}

\subsubsection{Cross-compiled RCC}
For cross-compilation, be sure to set \verb+OCPI_TARGET_PLATFORM+. At the end of the configuration, you should see something similar to this (\texttt{xilinx13\_3} shown):
\begin{lstlisting}
## ----------------------------------------- ##
## Configured to cross-compile for arm/x13_3 ##
## ----------------------------------------- ##
\end{lstlisting}

Once correctly \verb+./cross-configure+d, you can parallel make, e.g. \verb+make -j16 rcc+

\subsubsection{HDL}
If run from the \texttt{./hdl/} subdirectory, there is no change. If run from the top-level, then \verb+./ocpi-configure+ has to be run once to create the proper \texttt{Makefile}.

\section{FAQ}
\subsection{``No targets specified and no makefile found.  Stop.''}
\begin{itemize}
\item \verb+./ocpi-configure+
\item (or) \verb+./cross-configure+
\end{itemize}

\subsection{When to Reconfigure}
\begin{itemize}
\item Any time an environmental variable changes that would affect the make system's targeting.
\end{itemize}

\subsection{``/usr/bin/ld: skipping incompatible''}
This happens if the build system is confused and trying to combine x86 and ARM code. Try \verb+./ocpi-configure+, but if that doesn't work, \verb+make distclean+ first.

\subsection{``cannot create regular file /opt/opencpi/cdk/lib/''}
You are mixing and matching RPM-based usage (with an \verb+OCPI_CDK_DIR+ of \verb+/opt/opencpi/cdk+) and backwards-compatibility mode. \textbf{\textit{Don't.}}

\end{document}
